\documentclass{article}
\input{503macros.tex}

\begin{document}
\begin{center}
  {\LARGE CSC 503 Homework Assignment 9}\\[1pc]
  Out: October 27, 2016 \\
  Due: November 3, 2016 \\
  \unityid{aagrawa8}
\end{center}

\begin{enumerate}
\item \textbf{[20 points]} List all subformulas of the LTL
  formula, using the textbook binding priority conventions.
  \begin{displaymath}
    \NextX (q \Until (\neg \NextX p \implies r)) \WeakUntil 
    (\NextX \Sometime s \lor \Forever \NextX (r \Until \neg q))
  \end{displaymath}
  \begin{answer}
  \begin{itemize}
    \item $\NextX (q \Until (\neg \NextX p \implies r)) \WeakUntil 
    (\NextX \Sometime s \lor \Forever \NextX (r \Until \neg q))$
    \item $\NextX (q \Until (\neg \NextX p \implies r))$
    \item $q \Until (\neg \NextX p \implies r)$
    \item q
    \item $\neg \NextX p \implies r$
    \item $\NextX p$
    \item $\neg \NextX p$
    \item $p$
    \item $r$
    \item $\NextX \Sometime s \lor \Forever \NextX (r \Until \neg q)$
    \item $\NextX \Sometime s$
    \item $\Sometime s$
    \item $\Forever \NextX (r \Until \neg q)$
    \item $\NextX (r \Until \neg q)$
    \item $r \Until \neg q$ 
    \item $\neg q$


    \end{itemize}
  \end{answer}
  

The textbook states the satisfaction conditions for LTL statements are
as follows (Definition 3.6).
\begin{itemize}
\item $\pi \models \top$
\item $\pi \not\models \bot$
\item $\pi \models p$ iff $p \in L(s_1)$
  \begin{itemize}
  \item Regard $s_1$ as ``now'', the ``current state''
  \item Regard $\pi^2$ as its future
  \end{itemize}
\item $\pi \models \neg \phi$ iff $\pi \not\models \phi$
\item $\pi \models \phi \land \psi$ iff $\pi \models \phi$ and
  $\pi \models \psi$
\item $\pi \models \phi \lor \psi$ iff $\pi \models \phi$ or
  $\pi \models \psi$
\item $\pi \models \phi \implies \psi$ iff $\pi \models \psi$ whenever
  $\pi \models \phi$
\item $\pi \models \NextX \phi$ iff $\pi^2 \models \phi$
\item $\pi \models \Forever \phi$ iff $\pi^i \models \phi$ for all
  $i \ge 1$
\item $\pi \models \Sometime \phi$ iff $\pi^i \models \phi$ for some
  $i \ge 1$
\item $\pi \models \phi \Until \psi$ iff \\
  \qquad
  $\pi^i \models \psi$ for some $i \ge 1$ and $\pi^j \models
  \phi$ for all $1 \le j < i$
\item $\pi \models \phi \WeakUntil \psi$ iff either $\pi^k \models
  \phi$ for all $k \ge 1$; or \\
  \qquad
  $\pi^i \models \psi$ for some $i \ge 1$ and $\pi^j \models \phi$
  for all $1 \le j < i$
\item $\pi \models \phi \Release \psi$ iff either $\pi^k \models
  \phi$ for all $k \ge 1$; or \\
  \qquad
  $\pi^i \models \phi$ for some $i \ge 1$ and $\pi^j \models \psi$
  for all $1 \le j < i$
\end{itemize}

\item \textbf{[20 points]} Using these definitions, demonstrate
  that if $\pi \models \phi \WeakUntil \psi$, then
  $\pi \models (\phi \Until \psi) \lor \Forever \phi$.

\begin{answer}
 $\begin{array}{lll}$
1.$&$ $\pi \models \phi \WeakUntil \psi$ $&$Given\\
2. $&$ iff either $\pi^k \models \phi$ for all $k \ge 1$; or $&$ From the definition\\ 
3. $&$ 
  $\pi^i \models \psi$ for some $i \ge 1$ and $\pi^j \models \phi$
  for all $1 \le j < i$ $&$ From the definition\\
4. $&$ \item $\pi \models \Forever \phi$ $&$ Substituting i for k in 2 and from the definition of $\pi \models \Forever \phi$ \\
5. $&$ $\pi \models \phi \Until \psi$ $&$ From the definition of $\pi \models \phi \Until \psi$\\
6. $&$ $\pi \models (\phi \Until \psi) \lor \Forever \phi$ $&$ From the definition of $\pi \models \phi \lor \psi$ using on 5 and 4 \\
$\end{array} $ 
\end{answer}

\noindent
Answer the following questions concerning the truth of statements
about the following transition model $M$.
\begin{center}
  \begin{tikzpicture}[>=stealth',shorten >=1pt,auto,node
    distance=3cm]
    \node[state,label=-135:$s_1$] (s1) {$a b$};
    \node[state,label=-45:$s_2$] (s2) [right of=s1] {};
    \node[state,label=-135:$s_3$] (s3) [below of=s1] {$a$};
    \node[state,label=-45:$s_4$] (s4) [below of=s2] {$b$};
    \path[->] 
    (s1) edge  [loop left] node {} (s1)
    (s1) edge              node {} (s3)
    (s1) edge  [bend left] node {} (s2)
    (s2) edge  [bend left] node {} (s4)
    (s2) edge  [bend left] node {} (s1)
    (s3) edge  [loop left] node {} (s3)
    (s4) edge  [bend left] node {} (s2);
  \end{tikzpicture}
  \\
  {Model $M$}
\end{center}
Recall that all paths are infinite.  To indicate a path that ends with
a repeated set of states, put parenthesis around the repeated
subsequence (e.g., $(s_1, s_2)^\infty$.  To indicate a path in which
the initial subsequence $s_1, s_2$ is followed by any possible
continuing path, write ``$s_1, s_2,$ (any)''.


\item \textbf{[5 points]} Identify a path starting in state $s_1$
  that satisfies $\Forever a$ and explain why it does.
    \begin{answer}
     $\pi = (s_1, (s_3)^\infty)$ \\
     Since $s_3$ has a loop to itself, this path will forever be true containing literal a.
    \end{answer}
\item \textbf{[5 points]} Determine whether
  $M, s_1 \models \Forever a$ and explain why or why not.
 \begin{answer}
  No it wont be true for this path $\pi =  (s_1, (s_2 \implies s_4)^\infty)$ This path will not have literal a forever as it will keep looping back and forth between $s_2$ and $s_4$.
 \end{answer}
\item \textbf{[10 points]} Is there a state
  $s \in \{ s_1, s_2, s_3, s_4 \}$ such that
  $M, s \models \Forever a$?  If so, identify one; if not, explain why
  not.
 \begin{answer}
  $s_3$ is the state where $M, s \models \Forever a$. Since $s_3$ has only 1 transition to take, the path taken in future will always have literal a.
 \end{answer}
\item \textbf{[5 points]} Identify a path starting in state $s_1$
  that satisfies $b \Until a$ and explain why it does.
 \begin{answer}
  The path $\pi =  (s_1)^\infty $ will satisfy  $b \Until a$. Since it will always start from $s_1$ we will always have literal a.
 \end{answer}
\item \textbf{[5 points]} Determine whether $M, s_1 \models
  b \Until a$ and explain why or why not.
 \begin{answer}
  Yes this Model does. Since all path will start from $s_1$ which does contain literal a and we dont have to check further according to the definition of $\Until$.
 \end{answer}
\item \textbf{[5 points]} Identify a path starting in state $s_1$
  that satisfies $b \Until \Sometime \neg b$ and explain why it does.
 \begin{answer}
  The path $\pi =  (s_1 , (s_3)^\infty)$ will satisfy  $b \Until \Sometime \neg b$. Since at $s_1$ we have literal b and sometime in future it will not have literal b when it reaches $s_3$. And it will stay in $s_3$.
 \end{answer}
\item \textbf{[5 points]} Determine whether
  $M, s_1 \models b \Until \Sometime \neg b$ and explain why or why
  not.
 \begin{answer}
 No, M will not be true always. Consider the path $\pi =  (s_1,(s_2,s_4)^\infty)$. We will have literal b in next state and then sometime in future at $s_4$ it encountered literal b.
 \end{answer}
\item \textbf{[5 points]} Identify a path starting in state $s_1$ that
  satisfies $\NextX a \Until \NextX (\neg a \land \neg b)$ and explain
  why it does.
 \begin{answer}
  The path $\pi =  (s_1,s_1, s_2, (any))$  will satisfy  $\NextX a \Until \NextX (\neg a \land \neg b)$. At $s_1$ we have literal a until the next state transit to $s_2$ which is empty making $\neg a \land \neg b$ true and we dont have to check further according to the definition of $\Until$.
 \end{answer}
\item \textbf{[5 points]} Determine whether
  $M, s_1 \models \NextX a \Until \NextX (\neg a \land \neg
  b)$ and explain why or why not.
 \begin{answer}
  No it will not. Lets consider a path $\pi =  (s_1, s_1)^\infty$. In the next state $s_1$ it will make  $\neg a \land \neg b$ false since it contains both literals a and b.
 \end{answer}
\item \textbf{[5 points]} Identify a path starting in state $s_1$ that
  satisfies $\NextX (\neg a \Until (b \Until \Forever \neg a))$ and
  explain why it does.
 \begin{answer}
 The path $\pi = ( s_1, s_2, s_1, (s_3)^\infty)$ will satisfy $\NextX (\neg a \Until (b \Until \Forever \neg a))$. The next state of $s_2$ is true since it doesn't contain a, until you come back to state $s_1$ where b is true until you go to $s_3$ where it is always true since b is not in $s_3$.
 \end{answer}
\item \textbf{[5 points]} Determine whether
  $M, s_1 \models \NextX (\neg a \Until (b \Until \Forever \neg b))$
  and explain why or why not.
 \begin{answer}
  This is not true for a path $\pi =  (s_1)^\infty$ as it has encountered a in the next state itself.
 \end{answer}
\end{enumerate}

\end{document}
