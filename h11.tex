\documentclass{article}
\input{503macros.tex}

\begin{document}
\begin{center}
  {\LARGE CSC 503 Homework Assignment 11}\\[1pc]
  Out: November 10, 2016 \\
  Due: November 17, 2016 \\
  \unityid{aagrawa8}
\end{center}


See the macro instructions document for examples of how to write
program correctness proofs using the linear Fitch notation, such as
the proof in Example 4.13.1 on page 277:
\begin{displaymath}
  \begin{nd}
    \have{} {\Hcond{y = 5}}           \Precondition{}
    \have{} {\Hcond{y + 1 = 6}}       \Implied{}
    \code{x = y + 1;}
    \have{} {\Hcond{x = 6}}           \Assignment{}
  \end{nd}
\end{displaymath}

\begin{enumerate}

\item \textbf{[20 points]} Recall that the arithmetic expressions in
  the simple textbook programming language only refer to integers.
  Use the proof rule for assignment and integer arithmetic entailment
  as appropriate to show the validity of
  \begin{displaymath}
    \parturn \Hoare{y > z}{P}{y = 2z}.
  \end{displaymath}
  where $P$ is the program 
  \begin{displaymath}
    \begin{nd}
      \code{x = z;}
      \code{x = x + x - y;}
      \code{x = x + y + y;}
      \code{y = x - y;}
    \end{nd}
  \end{displaymath}

\begin{answer}
\begin{displaymath}
  \begin{nd}
    \have{} {\Hcond{y > z}}           \Precondition{}
    \have{} {\Hcond{z+z=2z}}       \Implied{}
    \code{x = z;}
    \have{} {\Hcond{x + x =2z}}           \Implied{}
    \have{} {\Hcond{x + x - y + y =2z}}           \Assignment{}
    \code{x = x + x - y;}
    \have{} {\Hcond{x + y =2z}}           \Implied{}
    \have{} {\Hcond{x + y + y-y=2z}}           \Assignment{}
    \code{x = x + y + y;}
    \have{} {\Hcond{x-y=2z}}           \Assignment{}
    \code{y = x - y;}
    \have{} {\Hcond{y=2z}}           \Assignment{}
  \end{nd}
\end{displaymath}
\end{answer}

\newcommand{\sign}{\text{sign}}

\item \textbf{[40 points]} Prove the validity of the sequent
  $\parturn \Hoare{x \neq 0 \land y \neq 0 \land x < y}{Q}{z =
    \sign(xy)}$ where $\sign(z)$ is 1 if $z$ is positive, -1 if $z$ is
  negative, and 0 if $z$ is 0, and where the code of $Q$ is given by
%  \begin{minipage}[t]{0.3\linewidth}
    \begin{displaymath}
      \begin{nd}
        \code{if (0 < x) \{}
        \open
        \code{z = 1;}
        \close
        \code{\} else \{}
        \open
        \code{if (0 < y) \{}
        \open
        \code{z = -1;}
        \close
        \code{\} else \{}
        \open
        \code{z = 1;}
        \close
        \code{\};}
        \close
        \code{\};}
      \end{nd}
    \end{displaymath}


\begin{answer}
    \begin{displaymath}
      \begin{nd}
      \have{} {\Hcond{x \neq 0 \land y \neq 0 \land x < y}} \Precondition{}
        \code{if (0 < x) \{}
        \open
        \have{} {\Hcond{x \neq 0 \land y \neq 0 \land x < y \land 0<x}} 
        \IfStatement{}
        \have{} {\Hcond{0 < y \land 0<x \land 0 < xy}} 
        \Implied{}
        \have{} {\Hcond{1 = sign(xy)}} 
        \Implied{}
        \code{z = 1;}
        \have{} {\Hcond{z = sign(xy)}} 
        \Assignment{}
        \close
        \code{\} else \{}
        \open
        \have{} {\Hcond{x \neq 0 \land y \neq 0 \land x < y \land 0 > x}} 
        \IfStatement{}
        \have{} {\Hcond{y \neq 0 \land x < y \land 0 > x}} 
        \Implied{}
        \code{if (0 < y) \{}
        \open
        \have{} {\Hcond{y \neq 0 \land x < y \land 0 > x \land 0 < y}} \IfStatement{}
        \have{} {\Hcond{0 > x \land 0 < y \land 0 > xy}} \Implied{}
        \have{} {\Hcond{-1 = sign(xy)}} 
        \Implied{}
        \code{z = -1;}
        \have{} {\Hcond{z = sign(xy)}} 
        \Assignment{}
        \close
        \code{\} else \{}
        \open
        \have{} {\Hcond{y \neq 0 \land x < y \land 0 > x \land 0 > y}} \IfStatement{}
        \have{} {\Hcond{0 > x \land 0 > y \land 0 < xy}} \Implied{}
        \have{} {\Hcond{1 = sign(xy)}} 
        \Implied{}
        \code{z = 1;}
        \have{} {\Hcond{z = sign(xy)}} 
        \Assignment{}
        \close
        \code{\};}
        \close
        \code{\};}
        \have{} {\Hcond{z = sign(xy)}} 
        \IfStatement{}
      \end{nd}
    \end{displaymath}
\end{answer}

\item \textbf{[40 points]} Prove the validity of the sequent
  $\parturn \Hoare{y \neq 0}{R}{w = x^y}$ where the code of $R$ is
  given by
  \begin{displaymath}
    \begin{nd}
      \code{w = 1;}
      \code{z = 0;}
      \code{while (z != y) \{}
      \open
      \code{w = w * x;}
      \code{z = z + 1;}
      \close
      \code{\};}
    \end{nd}
  \end{displaymath}


\begin{answer}
\begin{displaymath}
    \begin{nd}
      \have{} {\Hcond{y \neq 0}}           \Precondition{}
      \have{} {\Hcond{1 = x^0}}         \Implied{}
      \code{w = 1;}
      \have{} {\Hcond{w = x^0}}         \Assignment{}
      \code{z = 0;}
      \have{} {\Hcond{w = x^z}}         \Assignment{}
      \code{while (z != y) \{}
      \open
      \have{} {\Hcond{z \neq y \land w = x^z}}         \InvariantGuard{}
      \have{} {\Hcond{w * x = x^{z+1}}} \Implied{}
      \code{w = w * x;}
      \have{} {\Hcond{w = x^{z+1}}} \Assignment{}
      \code{z = z + 1;}
      \have{} {\Hcond{w = x^z}} \Assignment{}
      \close
      \code{\};}
      \have{} {\Hcond{\neg (z \neq y) \land w = x^z}} \PartialWhile{}
      \have{} {\Hcond{w = x^y}} \Implied{}
    \end{nd}
  \end{displaymath}
    
\end{answer}

\end{enumerate}

\end{document}
