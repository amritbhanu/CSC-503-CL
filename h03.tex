\documentclass{article}
\input{503macros.tex}

\begin{document}
\begin{center}
  {\LARGE CSC 503 Homework Assignment 3}\\[1pc]
  Out: August 30, 2016 \\
  Due: September 6, 2016 \\
  \unityid{aagrawa8}
\end{center}

\begin{enumerate}

\item \textbf{[30 points]} Using the method described in lecture,
  construct a formula $\phi$ in \textbf{DNF} to match the following
  truth table.  Show and explain any intermediate steps.
  \begin{center}
    \begin{tabular}{ccc|c}
      $p$ & $q$ & $r$ & $\phi$ \\ \hline
      T & T & T & T\\
      T & T & F & F\\
      T & F & T & F\\
      T & F & F & T\\
      F & T & T & F\\
      F & T & F & F\\
      F & F & T & T\\
      F & F & F & T
    \end{tabular}
  \end{center}
  
  \begin{answer}
  	\begin{enumerate}
  	  \item Rows with truth values '$T$' are $1, 4, 7, 8$.
  	  \item The formula $\phi$ can be represented in DNF as $(Row1 \lor Row4 \lor
  	  Row7 \lor Row8)$ which means that except for these there can be no other values of $T$.
  	  \item Each Row can be represented by the corresponding values of the
  	  literals. So the formula becomes $\phi : (p \land q \land r) \lor ( p
  	  \land \neg q\land \neg r) \lor (\neg p \land \neg q \land r) \lor (\neg p \land \neg q \land \neg r) $
	\end{enumerate}
	
	Thus the DNF for the given truth table is $(p \land q \land r) \lor ( p
  	  \land \neg q\land \neg r) \lor (\neg p \land \neg q \land r) \lor (\neg p \land \neg q \land \neg r)$.
  \end{answer}

\item \textbf{[30 points]} Using the method described in lecture,
  construct a formula $\phi$ in \textbf{CNF} to match the following
  truth table.  Show and explain any intermediate steps.
  \begin{center}
    \begin{tabular}{ccc|c}
      $p$ & $q$ & $r$ & $\phi$ \\ \hline
      T & T & T & T\\
      T & T & F & T\\
      T & F & T & F\\
      T & F & F & T\\
      F & T & T & F\\
      F & T & F & F\\
      F & F & T & F\\
      F & F & F & T
    \end{tabular}
  \end{center}
   \begin{answer}
  	\begin{enumerate}
  		\item Rows with truth values $F$ are $3, 5, 6, 7$
  		\item The formula $\phi$ can be represented in CNF as $(\neg Row3 \land
  		\neg Row5 \land \neg Row6 \land \neg Row7)$ which means that except for the
  		false value of the rows $3, 4, 6, 7$ others are $T$.
  		\item Each Row can be represented by the corresponding values of the
  		literals. So the formula becomes $\phi : (\neg (p \land \neg q \land r)
  		\land \neg(\neg p \land q \land r) \land \neg(\neg p \land q \land
  		\neg r) \land \neg(\neg p \land \neg q \land r))$
  		\item Reducing the above formula using De'Morgans, we get $\phi: (\neg
  		p \lor q \lor \neg r) \land ( p \lor \neg q \lor \neg r) \land (p \lor \neg q \lor r) \land (p \lor q \lor \neg r)$
  	\end{enumerate}
  	
  	Thus the CNF for the given truth table is $ (\neg
  		p \lor q \lor \neg r) \land ( p \lor \neg q \lor \neg r) \land (p \lor \neg q \lor r) \land (p \lor q \lor \neg r)$
  	\end{answer}

\item \textbf{[40 points]} Apply the following version of the
  algorithm HORN from pages 66--67 of the textbook to the following
  Horn formula $\varphi$.
  \begin{enumerate}
  \item Mark all occurrences of $\top$ in $\varphi$.
  \item Marked w,v from $(\top \implies w)$, $(\top \implies v)$.
  \begin{displaymath}
    \varphi = 
    \begin{array}{|lll}
      1.  & (\top \implies \textbf{w})                  & \land \\
      2.  & (w \implies q)                     & \land \\
      3.  & (x \land t \implies \bot)          & \land \\
      4.  & (q \land r \implies p)             & \land \\
      5.  & (v \implies s)                     & \land \\
      6.  & (w \implies r)                     & \land \\
      7.  & (r \land s \implies x)             & \land \\
      8.  & (\top \implies \textbf{v})                  & \land \\
      9.  & (v \land q \implies u)             & \land \\
      10. & (p \land r \land s \implies u)     & \land \\
      11. & (u \implies v)                     &       \\
    \end{array}
  \end{displaymath}
  \item Marked q,r,s from $(w \implies q)$, $(v \implies s)$, $(w \implies r)$.
  \begin{displaymath}
    \varphi = 
    \begin{array}{|lll}
      1.  & (\top \implies \textbf{w})                  & \land \\
      2.  & (w \implies \textbf{q})                     & \land \\
      3.  & (x \land t \implies \bot)          & \land \\
      4.  & (q \land r \implies p)             & \land \\
      5.  & (v \implies \textbf{s})                     & \land \\
      6.  & (w \implies \textbf{r})                     & \land \\
      7.  & (r \land s \implies x)             & \land \\
      8.  & (\top \implies \textbf{v})                  & \land \\
      9.  & (v \land q \implies u)             & \land \\
      10. & (p \land r \land s \implies u)     & \land \\
      11. & (u \implies v)                     &       \\
    \end{array}
  \end{displaymath}
  \item Marked p,x,u from $(q \land r \implies p)$, $(r \land s \implies x)$, $(v \land q \implies u)$.
  \begin{displaymath}
    \varphi = 
    \begin{array}{|lll}
      1.  & (\top \implies \textbf{w})                  & \land \\
      2.  & (w \implies \textbf{q})                     & \land \\
      3.  & (x \land t \implies \bot)          & \land \\
      4.  & (q \land r \implies \textbf{p})             & \land \\
      5.  & (v \implies \textbf{s})                     & \land \\
      6.  & (w \implies \textbf{r})                     & \land \\
      7.  & (r \land s \implies \textbf{x})             & \land \\
      8.  & (\top \implies \textbf{v})                  & \land \\
      9.  & (v \land q \implies \textbf{u})             & \land \\
      10. & (p \land r \land s \implies u)     & \land \\
      11. & (u \implies v)                     &       \\
    \end{array}
  \end{displaymath}

  \end{enumerate}
  
  The order of propositional letters marked is $w, v, q, r, p, x, u$. Since there is no conjunction to mark $t$, and hence $\bot$ is not marked. The $\varphi$ is Satisfiable using Horn Formula

\end{enumerate}
\end{document}
