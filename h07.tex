\documentclass{article}
\input{503macros.tex}

\begin{document}
\begin{center}
  {\LARGE CSC 503 Homework Assignment 7}\\
  Out: October 17, 2016 \\
  Due: October 25, 2016 \\
  \unityid{aagrawa8}
\end{center}


The first two problems refer to the substitutions
\begin{eqnarray*}
  \theta &=& \{ u/v, f(a)/w, g(f(x))/y \} \\
  \sigma &=& \{ v/w, a/u, g(y)/y, b/x \}.
\end{eqnarray*}

\begin{enumerate}

\item[1.] \textbf{[15 points]} Compute the product substitution
  $\theta\sigma$, showing how you obtained the result.
  \begin{answer}
  \begin{eqnarray*}
  \theta\sigma &=& \{ u\sigma/v, f(a)\sigma/w, g(f(x))\sigma/y, v/w, a/u, g(y)/y, b/x\} \\
  &=& \{ a/v, f(a)\sigma/w, g(f(x))\sigma/y, v/w, a/u, g(y)/y, b/x\}\\
  &=& \{ a/v, f(a)/w, g(f(x))\sigma/y, a/u, g(y)/y, b/x\}\\
  &=& \{ a/v, f(a)/w, g(f(b))/y, a/u, b/x\}
\end{eqnarray*}
  \end{answer}

\item[2.] \textbf{[15 points]} Compute the product substitution
  $\sigma\theta$.
  \begin{answer}
  \begin{eqnarray*}
  \sigma\theta &=& \{  v\theta/w, a\theta/u, g(y)\theta/y, b\theta/x, u/v, f(a)/w, g(f(x))/y\} \\
&=& \{  u/w, a\theta/u, g(y)\theta/y, b\theta/x, u/v, g(f(x))/y\}\\
&=& \{  u/w, a/u, g(y)\theta/y, b\theta/x, u/v, g(f(x))/y\}\\
&=& \{  u/w, a/u, g(g(f(x)))/y, b\theta/x, u/v\}\\
&=& \{  u/w, a/u, g(g(f(x)))/y, b/x, u/v\}
\end{eqnarray*}
  \end{answer}

\end{enumerate}


For your reference, the algorithm to calculate the most general
unifier $\sigma$ of a set $S$ of predicate expressions consists of the
following steps.
\begin{itemize}

\item Step 0:
  \begin{enumerate}
  \item Let $S_0 = S$
  \item Let $\sigma_0 = \epsilon$
  \end{enumerate}

\item Step $k+1$:
  \begin{enumerate}
  \item If $S_k$ has exactly one element, return the product
    substitution $\sigma_0\cdots\sigma_k$ as the value of $\sigma$
  \item If the disagreement set $D(S_k)$ contains both a variable
    $v$ and a term $t$ in which $v$ \emph{does not occur}, then
    \begin{enumerate}
    \item Choose least such pair (alphabetical order by variable, then
      by term)
    \item Let $\sigma_{k+1} = \{ t/v \}$
    \item Let $S_{k+1} = S_k \sigma_{k+1}$
    \item Proceed to step $k+2$
    \end{enumerate}
  \item Otherwise, announce that $S$ has no unifier
  \end{enumerate}
\end{itemize}

In the following two problems, apply the unification algorithm to each
of the following sets, assuming that $a$ is a constant symbol, $f$ is
a function symbol, $P$ is a predicate symbol, and $x,y,z$ are variable
symbols.  For each set, at each step $k$, show (1) the disagreement
set $D(S_k)$ of $S_k$, (2) the substitution $\sigma_k$ if there is
one, or an explanation why there is no unifying substitution, and (3)
the result $S_{k+1}$ of applying $\sigma_k$ to $S_k$.  If the set
unifies, show also (4) the overall substitution
$\sigma_0 \dots \sigma_k$ expressed as a single substitution (that is,
compute the product of the stepwise substitutions).

\begin{enumerate}

\item[3.] \textbf{[15 points]}
  $S = \{ P(x,y,z), P(a,f(x,z),f(x,y)) \}$
\begin{answer}
        Initializing $\sigma_0$ to $\{\}$
		
		$S_0 = \{ P(x,y,z), P(a,f(x,z),f(x,y)) \}$
		
		$D(S_0) = \{x, a\}$
		\bigskip
		
		$\sigma_1 = \{a/x\}$
		
		$S_1 = \{ P(a,y,z), P(a,f(a,z),f(a,y)) \}$
		
		$D(S_1) = \{y, f(a,z)\}$
		\bigskip
		
		$\sigma_2 = \{f(a,z)/y\}$
		
		$S_2 = \{ P(a,f(a,z),z), P(a,f(a,z),f(a,f(a,z))) \}$
		
		$D(S_2) = \{z, f(a,f(a,z))\}$
		Unification is not possible as now there is no substitution for z as it is a bound variable.
\end{answer}

\item[4.] \textbf{[15 points]}
  $S = \{ P(f(a),y), P(x,a) \}$
  \begin{answer}
        Initializing $\sigma_0$ to $\{\}$
		
		$S_0 = \{ P(f(a),y), P(x,a) \}$
		
		$D(S_0) = \{f(a), x\}$
		\bigskip
		
		$\sigma_1 = \{f(a)/x\}$
		
		$S_1 = \{ P(f(a),y), P(f(a),a) \}$
		
		$D(S_1) = \{y, a\}$
		\bigskip
		
		$\sigma_2 = \{a/y\}$
		
		$S_2 = \{ P(f(a),a), P(f(a),a) \}$
		
		$|S_2| = 1$		
		\bigskip
		
		$\sigma = \sigma_0 \cdot{} \sigma_1 \cdot{} \sigma_2$
		
		$\sigma = \{\} \cdot{} \{f(a)/x\} \cdot{} \{a/y\}$
		
		$\sigma = \{f(a)/x\} \cdot{} \{a/y\}$
		
		$\sigma = \{f(a)/x, a/y\}$
		
		Unification is feasible for above $\sigma$.
\end{answer}
\end{enumerate}


% \vfill
% \begin{center}
%   (Assignment continued next page)
% \end{center}

\newpage

The following is an example of a propositional resolution refutation
proof of $Q$ from premises $P$ and $P \implies Q$, with the proof
written in linear form.
\begin{center}
  \begin{tabular}{lll}
    Line & Clause & Justification \\ \hline
    1. & $\{ P \}$ & Given \\
    2. & $\{ \neg P, Q \}$ & Given \\ \hline
    3. & $\{ \neg Q \}$ & Refutation clause, negation of goal $Q$ \\
    4. & $\{ \neg P \}$ & Resolving (2) with (3) on $Q$  \\
    5. & $\{ \}$ & Resolving (4) with (1) on $P$  \\ 
  \end{tabular}
\end{center}
If the negation of the goal had consisted of several clauses, each of
those would have been listed on a separate line as a refutation clause
(as is done for the single clause on line 3).  The order in which
lines are mentioned in lines 4 and 5 does not matter.

The following is an example of a predicate resolution refutation proof
of $Q(a)$ from premises $P(a)$ and $\forall x\ [P(x) \implies Q(x)]$,
again with the proof written in linear form.
\begin{center}
  \begin{tabular}{llll}
    Line & Clause & Justification & Unifier \\ \hline
    1. & $\{ P(a) \}$ & Given & \\
    2. & $\{ \neg P(x_2), Q(x_2) \}$ & Given & \\ \hline
    3. & $\{ \neg Q(a) \}$ & Refutation clause, negation of goal
                             $Q(a)$ & \\
    4. & $\{ \neg P(a) \}$ & Resolving (2) with (3) on $Q$ & $\{ a/x_2
                                                             \}$ \\
    5. & $\{ \}$ & Resolving (4) with (1) on $P$ & $\{ \}$  \\ 
  \end{tabular}
\end{center}
Here the variables in line 2 have been standardized apart, which was
not really necessary for this proof as line 2 represents the only
given involving variables.  If the resolution steps in lines 4 and 5
had involved multiple literals in one of the clauses, the
justification would have needed to indicate the specific literals
involved.

In the following two problems concerning resolution proofs, write the
proofs in either linear or tree form.

\begin{enumerate}
    
\item[5.] \textbf{[20 points]} Let $P,Q,R,S$ be atomic propositional
  symbols, and give a resolution refutation proof of $\neg R$ in
  either tree or linear form from the following clauses.
  \begin{answer}
          
  
  \begin{center}
    \begin{tabular}{lll}
      Line & Clause & Justification \\ \hline
      1. & $\{ P, Q, \neg R \}$ & Given \\
      2. & $\{ \neg P, S \}$ & Given \\
      3. & $\{ \neg S \}$ & Given \\
      4. & $\{ \neg Q \}$ & Given \\ \hline
      5. & $\{ R \}$ & Refutation clause, negation of goal. \\
      6. & $\{ P, Q \}$ & Resolving (1) with (5) on $R$. \\
      7. & $\{ P \}$ & Resolving (6) with (4) on $Q$.\\
      8. & $\{ \neg P \}$ & Resolving (2) with (3) on $S$ \\
      9. & $\{ \}$ & Resolving (7) with (8) on $P$. 
    \end{tabular}
  \end{center}
  We proved $\neg R$ by deriving empty clause by giving a resolution refutation of the goal.
\end{answer}
  
\item[6.] \textbf{[20 points]} Let $a$ be a constant symbol, $f$ be a
  function symbol, $P$ a ternary predicate symbol, and $x,y,z$ be
  variables.  Rewrite the following three clauses with the variables
  standardized apart and then give a resolution refutation of the set
  of the three rewritten clauses.  At each step of the refutation,
  indicate the literals being resolved together and the most general
  unifying substitution used in the proof.  If there is only one
  possible choice of complementary literals, you need indicate only
  the predicate symbol being resolved on, not the arguments of the
  literals.
  
  \begin{center}
    \begin{tabular}{llll}
      Line & Clause & Justification & Substitution \\ \hline
      1. & $\{ P(a,x,x) \}$ & Given & \\
      2. & $\{ \neg P(x,y,z), P(f(f(x)),y,f(z)) \}$ & Given & \\
      3. & $\{ \neg P(f(f(a)),a,f(a)) \}$ & Given & \\ \hline
    \end{tabular}
  \end{center}
  
  \begin{answer}
    \begin{center}
    \begin{tabular}{llll}
      Line & Clause & Justification & Substitution \\ \hline
      1. & $\{ P(a,x_1,x_1) \}$ & Given & \\
      2. & $\{ \neg P(x_2,y_2,z_2), P(f(f(x_2)),y_2,f(z_2)) \}$ & Given & \\
      3. & $\{ \neg P(f(f(a)),a,f(a)) \}$ & Given & \\ \hline
      4. & $\{P(f(f(a)),a,f(a))\}$ & Resolving (2) with (1)  & $\{a/x_1,a/x_2,a/y_2,a/z_2\}$\\
      5. & $\{\}$ & Resolving (4) with (3) & $\{\}$
    \end{tabular}
  \end{center}
  \end{answer}

\end{enumerate}

\end{document}
